\section{Introduction}
\label{sec:introduction}
Today's cryptography for modern communication systems are mostly based on asymmetric cryptography algorithm, known as public-key cryptography.\\ One of the widely used public-key cryptography algorithm is RSA\cite{jonsson_pkcs_nodate}. It is based on the computational hardness assumption of integer factorisation. However, in the early 2000s, it has been proven that such systems are vulnerable to quantum computer.
Thus a new mathematical model should be found that could be proved quantum resistant.\cite{bernstein_post-quantum_2009-1} \\
The NTRU crypto-system\cite{schanck_practical_2015} (described in section \ref{sec:background}) is such an alternative based on the closest lattice vector problem.\\
Asymmetric cryptography is also an important part of the concept of Internet of Things (IoT). \\In fact the challenge for such hardware is the possibility for secure communication while keeping the cost of the hardware low. Symmetric cryptography is already widely used and proven to be performant and cost efficient for such small controller and limited hardware. But symmetric cryptography do provide a low security scheme.\\
The NTRU public-key crypto-system main characteristics are the low memory and computational requirements while keeping a high security level in communication thanks to the asymmetric scheme of encryption. This makes the NTRU crypto-system a good candidates for being implemented on IoT hardware (e.g. 8 bit AVR controller).\\
\\
In this work we will go trough the different steps of implementation and performance measurement of  the NTRUencrypt algorithm on an 8 bit AVR controller. Specially for the mask generating utility function used to provide of a variable length primitive output. And if such implementation can approximate the performance of a standard 64 bit computer. 